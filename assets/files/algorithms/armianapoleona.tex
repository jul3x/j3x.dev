
% opcjonalne (obowiazkowe) parametry
\pagestyle{fancy}
\konkurs{Konkurs}
\etap{etap 1}
\day{1}
\date{01.01.1970}
%\setlogo{oiglogo}%brak ustawienia wstawia domyslnie logo talentu
%\setstopka{stopka_bw}
\title{\mbox{Armia Napoleona}}
\id{armianapoleona}
\RAM{64}

\begin{document}

\begin{tasktext}%
    \noindent
    Napoleon Bonaparte ma duży problem z ustawieniem swojej armii w rzędzie.
    Piechota jest podzielona na dwa typy żołnierzy - karabinierów i grenadierów.
    Efektywne ustawienie armii jest związane z dobrym wymieszaniem żołnierzy różnego typu z powodów taktycznych.
    Formacja jest poprawna tylko wtedy, gdy żaden grenadier nie sąsiaduje z drugim.
    Napoleon zastanawia się, ile jest możliwych poprawnych kombinacji ustawień dla danych liczb żołnierzy obu typów.
    Jako jego zaprzyjaźniony informatyk Twoim zadaniem jest podanie tej liczby.
	
    \section{Wejście}
	W pierwszym wejściu standardowego wejścia znajduje się liczba przypadków testowych $t$ ($1 \leqslant t \leqslant 10^{5}$).
	W każdym z kolejnych $t$ wierszy mamy daną liczbę $n$ ($1 \leqslant n \leqslant 10^6$) oznaczającą liczbę grenadierów oraz liczbę $k$ ($1 \leqslant k \leqslant 10^6$) będącą liczbą karabinierów.

	\section{Wyjście}
	W każdym z $t$ wierszy standardowego wyjścia należy wypisać liczbę poprawnych ustawień dla danego $n$ i $k$.
	Wynik powinien być podany modulo $10^9+7$.
	
	\oigprzyklady
\end{tasktext}
\end{document}
